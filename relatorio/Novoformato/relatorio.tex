%% BioMed_Central_Tex_Template_v1.06
%%                                      %
%  bmc_article.tex            ver: 1.06 %
%                                       %

%%IMPORTANT: do not delete the first line of this template
%%It must be present to enable the BMC Submission system to 
%%recognise this template!!

%%%%%%%%%%%%%%%%%%%%%%%%%%%%%%%%%%%%%%%%%
%%                                     %%
%%  LaTeX template for BioMed Central  %%
%%     journal article submissions     %%
%%                                     %%
%%         <14 August 2007>            %%
%%                                     %%
%%                                     %%
%% Uses:                               %%
%% cite.sty, url.sty, bmc_article.cls  %%
%% ifthen.sty. multicol.sty		   %%
%%				      	   %%
%%                                     %%
%%%%%%%%%%%%%%%%%%%%%%%%%%%%%%%%%%%%%%%%%


%%%%%%%%%%%%%%%%%%%%%%%%%%%%%%%%%%%%%%%%%%%%%%%%%%%%%%%%%%%%%%%%%%%%%
%%                                                                 %%	
%% For instructions on how to fill out this Tex template           %%
%% document please refer to Readme.pdf and the instructions for    %%
%% authors page on the biomed central website                      %%
%% http://www.biomedcentral.com/info/authors/                      %%
%%                                                                 %%
%% Please do not use \input{...} to include other tex files.       %%
%% Submit your LaTeX manuscript as one .tex document.              %%
%%                                                                 %%
%% All additional figures and files should be attached             %%
%% separately and not embedded in the \TeX\ document itself.       %%
%%                                                                 %%
%% BioMed Central currently use the MikTex distribution of         %%
%% TeX for Windows) of TeX and LaTeX.  This is available from      %%
%% http://www.miktex.org                                           %%
%%                                                                 %%
%%%%%%%%%%%%%%%%%%%%%%%%%%%%%%%%%%%%%%%%%%%%%%%%%%%%%%%%%%%%%%%%%%%%%


\NeedsTeXFormat{LaTeX2e}[1995/12/01]
\documentclass[10pt]{bmc_article}    



% Load packages
\usepackage{cite} % Make references as [1-4], not [1,2,3,4]
\usepackage{url}  % Formatting web addresses  
\usepackage{ifthen}  % Conditional 
\usepackage{multicol}   %Columns
\usepackage[utf8]{inputenc} %unicode support
\usepackage{listings}
\usepackage{graphicx}
%\usepackage[applemac]{inputenc} %applemac support if unicode package fails
%\usepackage[latin1]{inputenc} %UNIX support if unicode package fails
\usepackage{amssymb,amsmath}
\usepackage{listings}
\usepackage{color}
\usepackage{framed}
\usepackage[brazil]{babel}
\urlstyle{rm}
 
 
%%%%%%%%%%%%%%%%%%%%%%%%%%%%%%%%%%%%%%%%%%%%%%%%%	
%%                                             %%
%%  If you wish to display your graphics for   %%
%%  your own use using includegraphic or       %%
%%  includegraphics, then comment out the      %%
%%  following two lines of code.               %%   
%%  NB: These line *must* be included when     %%
%%  submitting to BMC.                         %% 
%%  All figure files must be submitted as      %%
%%  separate graphics through the BMC          %%
%%  submission process, not included in the    %% 
%%  submitted article.                         %% 
%%                                             %%
%%%%%%%%%%%%%%%%%%%%%%%%%%%%%%%%%%%%%%%%%%%%%%%%%                     


%\def\includegraphic{}
%\def\includegraphics{}



\setlength{\topmargin}{0.0cm}
\setlength{\textheight}{21.5cm}
\setlength{\oddsidemargin}{0cm} 
\setlength{\textwidth}{16.5cm}
\setlength{\columnsep}{0.6cm}

\newboolean{publ}

%%%%%%%%%%%%%%%%%%%%%%%%%%%%%%%%%%%%%%%%%%%%%%%%%%
%%                                              %%
%% You may change the following style settings  %%
%% Should you wish to format your article       %%
%% in a publication style for printing out and  %%
%% sharing with colleagues, but ensure that     %%
%% before submitting to BMC that the style is   %%
%% returned to the Review style setting.        %%
%%                                              %%
%%%%%%%%%%%%%%%%%%%%%%%%%%%%%%%%%%%%%%%%%%%%%%%%%%
 

%Review style settings
%\newenvironment{bmcformat}{\begin{raggedright}\baselineskip20pt\sloppy\setboolean{publ}{false}}{\end{raggedright}\baselineskip20pt\sloppy}

%Publication style settings
%\newenvironment{bmcformat}{\fussy\setboolean{publ}{true}}{\fussy}

%New style setting
\newenvironment{bmcformat}{\baselineskip20pt\sloppy\setboolean{publ}{false}}{\baselineskip20pt\sloppy}

% Begin ...
\begin{document}
\begin{bmcformat}


%%%%%%%%%%%%%%%%%%%%%%%%%%%%%%%%%%%%%%%%%%%%%%
%%                                          %%
%% Enter the title of your article here     %%
%%                                          %%
%%%%%%%%%%%%%%%%%%%%%%%%%%%%%%%%%%%%%%%%%%%%%%

\title{On the computational modeling of the innate immune system}
 
%%%%%%%%%%%%%%%%%%%%%%%%%%%%%%%%%%%%%%%%%%%%%%
%%                                          %%
%% Enter the authors here                   %%
%%                                          %%
%% Ensure \and is entered between all but   %%
%% the last two authors. This will be       %%
%% replaced by a comma in the final article %%
%%                                          %%
%% Ensure there are no trailing spaces at   %% 
%% the ends of the lines                    %%     	
%%                                          %%
%%%%%%%%%%%%%%%%%%%%%%%%%%%%%%%%%%%%%%%%%%%%%%

\author{Alexandre Bittencourt Pigozzo\correspondingauthor$^1$%
         \email{A B Pigozzo\correspondingauthor - alexbprr@gmail.com}
,
         Gilson Costa Macedo$^2$%
         \email{G C Macedo - gilson.macedo@ufjf.edu.br}%
 ,
         Rodrigo Weber dos Santos$^1$%
         \email{R W dos Santos - rodrigo.weber@ufjf.edu.br}
  ,
         Marcelo Lobosco$^1$%
         \email{M Lobosco - marcelo.lobosco@ufjf.edu.br}
      }
      

%%%%%%%%%%%%%%%%%%%%%%%%%%%%%%%%%%%%%%%%%%%%%%
%%                                          %%
%% Enter the authors' addresses here        %%
%%                                          %%
%%%%%%%%%%%%%%%%%%%%%%%%%%%%%%%%%%%%%%%%%%%%%%

%\address{%
%    \iid(1)Department of Zoology, Cambridge, Waterloo Road, London, UK\\
%    \iid(2)Marine Ecology Department, Institute of Marine Sciences Kiel, %
%        D\"{u}sternbrooker Weg 20, 24105 Kiel, Germany
%}%


\address{%
    \iid(1)Graduate Program in Computational Modeling, UFJF, Rua Jos\'{e} Louren\c{c}o Kelmer, s/n - Campus Universit\'{a}rio,
        Bairro S\~{a}o Pedro, CEP: 36036-900, Juiz de Fora, MG, Brazil\\
    \iid(2)Graduate Program in Biological Sciences, UFJF, Rua Jos\'{e} Louren\c{c}o Kelmer, s/n - Campus Universit\'{a}rio,
        Bairro S\~{a}o Pedro, CEP: 36036-900, Juiz de Fora, MG, Brazil
}%

\maketitle

%%%%%%%%%%%%%%%%%%%%%%%%%%%%%%%%%%%%%%%%%%%%%%
%%                                          %%
%% The Abstract begins here                 %%
%%                                          %%  
%% Please refer to the Instructions for     %%
%% authors on http://www.biomedcentral.com  %%
%% and include the section headings         %%
%% accordingly for your article type.       %%   
%%                                          %%
%%%%%%%%%%%%%%%%%%%%%%%%%%%%%%%%%%%%%%%%%%%%%%


\begin{abstract}
        % Do not use inserted blank lines (ie \\) until main body of text.
In recent years, there has been an increasing interest in the mathematical and computational modeling of the human immune system (HIS). 
Computational models of HIS dynamics may contribute to a better understanding of the relationship between complex phenomena and immune 
response; in addition, computational models will support the development of new drugs and therapies for different diseases. However, 
modeling the HIS is an extremely difficult task that demands a huge amount of work to be performed by multidisciplinary teams. In this 
study, our objective is to model the spatio-temporal dynamics of representative cells and molecules of the HIS during an immune response 
after the injection of lipopolysaccharide (LPS) into a section of tissue. LPS constitutes the cellular wall of Gram-negative bacteria, 
and it is a highly immunogenic molecule, which means that it has a remarkable capacity to elicit strong immune responses. We 
present a descriptive, mechanistic and deterministic model that is based on partial differential equations (PDE). Therefore, this model 
enables the understanding of how the different complex phenomena interact with structures and elements during an immune response. In 
addition, the model's parameters reflect physiological features of the system, which makes the model appropriate for general use.
\end{abstract}



\ifthenelse{\boolean{publ}}{\begin{multicols}{2}}{}




%%%%%%%%%%%%%%%%%%%%%%%%%%%%%%%%%%%%%%%%%%%%%%
%%                                          %%
%% The Main Body begins here                %%
%%                                          %%
%% Please refer to the instructions for     %%
%% authors on:                              %%
%% http://www.biomedcentral.com/info/authors%%
%% and include the section headings         %%
%% accordingly for your article type.       %% 
%%                                          %%
%% See the Results and Discussion section   %%
%% for details on how to create sub-sections%%
%%                                          %%
%% use \cite{...} to cite references        %%
%%  \cite{koon} and                         %%
%%  \cite{oreg,khar,zvai,xjon,schn,pond}    %%
%%  \nocite{smith,marg,hunn,advi,koha,mouse}%%
%%                                          %%
%%%%%%%%%%%%%%%%%%%%%%%%%%%%%%%%%%%%%%%%%%%%%%




%%%%%%%%%%%%%%%%
%% Background %%
%%
\section*{Introdução}

O entendimento de fenômenos complexos na Imunologia é um grande desafio para cientistas em vários campos 
		da biologia, medicina e farmacologia \cite{Carvalho2015}. Os fenômenos biológicos relacionados às infecções e ao sistema imune 
		podem ser descritos por diversos tipos de modelos dentre os quais destaca-se: equações diferenciais ordinárias, 
		equações diferenciais parciais, modelos baseados em agentes, autômatos celulares, dinâmica de sistemas, 
		redes de Petri, entre outros. Todos esses modelos podem contribuir com o entendimento de mecanismos presentes 
		na interação entre o patógeno e o hospedeiro através da modelagem e simulação de diversos aspectos das 
		respostas imunes \cite{Carvalho2015} [outras referencias].  
		
		Para muitos patógenos bacterianos, o sistema imunológico do hospedeiro elimina completamente as bactérias invasoras 
		e a infecção é resolvida, ou seja, a inflamação e os danos teciduais cessam. No entanto, em alguns casos, devido 
		a vários fatores como doenças pré-existentes, imunidade baixa, entre outros, as bactérias evadem o sistema imune 
		e estabelecem um ambiente favorável para a persistência caracterizando uma infecção crônica. 
		Em certos casos, essas infecções persistentes são assintomáticas 
		por longos períodos de tempo, mas podem sofrer reativação futura em doença clinicamente significativa, ou podem estar 
		associadas a malignidade ou disseminação subsequente da doença \cite{Grant2013}. Alternativamente, algumas infecções persistentes 
		resultam em sintomas crônicos clinicamente aparentes. Nesses casos, mesmo o tratamento padrão com antibióticos muitas 
		vezes não consegue esterilizar eficazmente infecções persistentes, e cursos prolongados ou repetidos de antibióticos 
		são necessários para a erradicação bem sucedida. Ao extremo, a supressão crônica ao longo da vida com antibióticos 
		pode ser necessária na ausência de erradicação \cite{Grant2013}.		
		
		Muitos fatores contribuem para a capacidade dos patógenos em estabelecer infecções persistentes, incluindo tanto 
		fatores hospedeiros quanto bacterianos. Certos agentes patogênicos parecem unicamente adaptados para escapar do 
		sistema imune do hospedeiro e persistir em indivíduos infectados por décadas na ausência de sintomas como, por exemplo, 
		\textit{Mycobacterium tuberculosis} ou \textit{Salmonella Typhi}. Outros patógenos como \textit{Pseudomonas aeruginosa} 
		ou \textit{Escherichia coli} podem causar tanto sintomas agudos como crônicos, com alterações específicas no hospedeiro, 
		facilitando o estabelecimento de uma infecção persistente \cite{Grant2013}.
		%Bacteria adapt to environmental stresses imposed by the host by entering a different physiologic state. 
		%A key element of this different physiologic state is a non-replicating or slowly replicating growth rate, 
		%which may have the additional benefit of contributing to a pathogen’s defense against antibiotic exposure. 

		Nos últimos anos, Redes de Petri estão sendo usadas para modelagem de sistemas biológicos. Exemplos disso incluem 
		a modelagem de reações bioquímicas, de redes de interação proteica, de redes de regulação gênica, 
		modelagem do sistema imunológico, entre outros. No contexto da modelagem do sistema imunológico, destaca-se o trabalho de Rafael \textit{et al.} ...
		
		%Objetivo	
		Neste trabalho, redes de Petri são utilizadas para estudar comportamentos observados em infecções bacterianas crônicas. 
		O objetivo deste trabalho é estudar o papel regulador dos macrófagos e a influência do dano tecidual durante a resposta imune 
		a uma infecção bacteriana crônica. 
				
		%Chronic conditions exhibit many of the same immune signatures of acute infection, including loss of barrier function, tissue destruction, and functional impairment, and are becoming increasingly prevalent worldwide (Karin et al., 2006)
		Uma condição crônica pode ser causada por uma infecção aguda como mostra o trabalho de Moraes \textit{et al} \cite{Fonseca2015} que ao estudar infecções com 
		\textit{Yesrsinia pseudotuberculosis} observaram que a infecção aguda teve um impacto de longa duração na estrutura e função do 
		sistema imune do intestino \cite{Fonseca2015}. 
		
		Em muitos casos, infecções crônicas estão associadas com uma inflamação crônica. A inflamação crônica é caracterizada por repetidos ciclos de migração 
		de células do sistema imune, produção de substâncias pró-inflamatórias e dano tecidual que se não forem contidos pelo próprio sistema imune podem levar a 
		consequências graves para o paciente. Os macrófagos são uma das principais células responsáveis pela regulação dessa resposta pró-inflamatória através, 
		por exemplo, da produção de citocinas anti-inflamatórias, da fagocitose de células apoptóticas e de células de tecido mortas, entre outras ações. 
		Portanto, caso o papel regulador dos macrófagos esteja prejudicado por algum motivo, o resultado pode ser fatal. 
		Nesse caso, os macrófagos poderiam contribuir para a continuação da inflamação e do dano tecidual. 
		
		%O problema com a inflamação não é com que frequência ela começa, mas com que frequência ela não diminui. Talvez nenhum fenômeno isolado contribua mais para a carga médica nas sociedades industrializadas do que a inflamação não resolvida. A inflamação não-resolvida não é uma causa primária de aterosclerose, obesidade, câncer, doença pulmonar obstrutiva crônica, asma, doença inflamatória intestinal, doença neurodegenerativa, esclerose múltipla ou artrite reumatóide, mas contribui significativamente para sua patogênese.

		  %Esta revisão ilustrativa, mas não abrangente, começa descrevendo diferentes formas de inflamação não-resolvida. Discutimos então o dano dos tecidos não infectados como propagador da inflamação. Esta é uma questão conceitual crítica, porque os estímulos microbianos não estão implicados em muitas formas de inflamação crônica, e é instrutivo avaliar o valor adaptativo de nossa capacidade de lançar um processo de dano tecidual desencadeado pelo próprio dano tecidual.
		
		%Fibrosis sufficient to interfere with organ function is a major medical problem after inflammation of arteries caused by accumulation of cholesterol, inflammation of the liver caused by viruses, alcohol, toxins or schistosome infections, inflammation of the lung associated with asthma or radiotherapy, and inflammation of the bowel in Crohn's disease, where fibrotic strictures (occlusions) often require surgery. Fibrosis arises from the excessive number, activity, or life span of collagen producing cells
				
		%Dados esses fatos, nos últimos anos, tem aumentado o interesse em entender como distintas populações de macrófagos contribuem para uma inflamação (Han et al., 2013; Xu et al., 2012; Xuet al., 2015)
										
		%Como objetivos específicos, destaca-se: 
		%- A investigação do papel dos macrófagos tanto como iniciador como importante regulador da resposta imune; 
		%- 		
		
		% Os modelos desenvolvidos neste trabalho se baseiam nas seguintes hipóteses: 		
% 		Hipóteses: 
% 		- Os macrófagos evitam um estado de inflamação persistente que poderia ser devastador para o tecido infectado. 
% 		Eles fazem isso controlando a resposta pró-inflamatória ... 
% 		- O dano tecidual promove a inflamação podendo causar uma resposta exarcebada do sistema imune e, como 
% 		consequência, mais inflamação. 

\section*{Biological background}\label{background}
%modificacoes: troquei proporcionate por provides

``Human body surfaces are protected by epithelia, which provide a physical barrier between internal and external environments. 
Epithelia make up the skin and lining of the tubular structures of the body (i.e., the gastrointestinal, respiratory and genitourinary 
tracts), and they form an effective barrier against the external environment. At the same time, epithelia can be crossed or settled by 
pathogens, causing infections. After crossing the epithelium, the pathogens encounter cells and molecules of the innate immune system, 
which immediately develop a response'' \cite{janeway}.

The body's initial response to an acute biological stress, such as a bacterial infection, is an acute inflammatory 
response \cite{janeway}. The strategy of the HIS is to keep some resident macrophages on guard in tissues to look for any signal of 
infection. When they find such a signal, the macrophages alert neutrophils (also known as polymorphonuclear neutrophils (PMNs)) 
that their help is required. Because of this communication, the cooperation between macrophages and neutrophils is essential to 
mount an effective defense against disease. Without macrophages to herd neutrophils toward the location of infection, the latter 
would circulate indefinitely in the blood vessels, impairing the control of systemic infections \cite{Sompayrac2008}.

The inflammation of an infectious tissue has many benefits for the control of the infection. In addition to recruiting cells and 
molecules of innate immunity from blood vessels to the location of the infected tissue, inflammation increases the lymph flux, 
which contains microorganisms and cells that carry antigens to neighboring lymphoid tissues; there, these cells will present the 
antigens to the lymphocytes and initiate the adaptive response. Once the adaptive response has been activated, the inflammation 
also shuttles the effector cells of the adaptive immune system to the location of infection \cite{janeway}.

A component of the cellular wall of Gram-negative bacteria, such as LPS, can trigger an inflammatory response through the interaction 
with receptors on the surface of some cells \cite{Sompayrac2008}. For example, the macrophages that reside in tissue recognize a bacterium through the 
binding of TLR4 (Toll-like receptor 4) with LPS. When receptors on the surface of macrophages bind to LPS, the macrophage starts 
to phagocytose, internally weakening the bacterium and secreting proteins known as cytokines and chemokines, as well as other molecules. 

In many inflammatory conditions, neutrophils dominate the initial influx of leukocytes into the inflamed tissue. The first wave of 
extravasated neutrophils is soon replaced by a second wave of monocytes \cite{Sompayrac2008}. A study presented initial proofs of the existence of this 
sequence of events \cite{Rebuck1955}. In that study, neutrophils dominated the leukocyte extravasation three hours after the beginning 
of the inflammation, and some time later, the extravasated cells were predominantly monocytes \cite{Rebuck1955}. 

The resolution of the inflammatory response is a complex process that includes the production of anti-inflammatory mediators and the 
apoptosis (or programmed death) of effector cells of the HIS, such as neutrophils \cite{Opal2000}. Anti-inflammatory cytokines form a set of 
immunoregulatory molecules that control the inflammatory response. These cytokines work together with specific inhibitors and cytokines' 
soluble receptors to regulate the immune response \cite{Opal2000}. A previous work \cite{Opal2000} demonstrated the participation of cytokines in 
inflammatory states. Primary anti-inflammatory cytokines include the antagonist receptor of IL-1 (Interleukin 1) in addition to IL-4, 
IL-6, IL-10, IL-11 and IL-13\cite{Opal2000}. Specifically, IL-10 is a strong inhibitor of many pro-inflammatory cytokines \cite{Fiorentino1991}, 
including IL-8 and TNF-$\alpha$ (tumor necrosis factor $\alpha$), which are produced both by monocytes \cite{Waal1991} and by 
neutrophils \cite{Cassatella1993,Marie1996}.

Apoptotic cells maintain membrane integrity for a small period of time and therefore need to be quickly removed to prevent a secondary 
necrosis and the consequent release of cytotoxic molecules, which cause inflammation and tissue damage \cite{Kennedy2009}. As a 
consequence of the phagocytosis of apoptotic cells by macrophages or dendritic cells, these phagocytic cells produce anti-inflammatory 
cytokines. For example, macrophages secrete TGF-$\beta$ (transforming growth factor $\beta$), which prevents the release of 
pro-inflammatory cytokines induced by LPS \cite{Lucas2006}. Additionally, the binding of apoptotic cells to macrophage receptor 
CD36 (cluster of differentiation 36) inhibits the production of pro-inflammatory cytokines such as TNF-$\alpha$, IL-1$\beta$ and IL-12; 
this binding also increases the secretion of TGF-$\beta$ and IL-10\cite{Voll1997}. 

\section*{Related work}\label{RelatedWorks}

This section presents and discusses other models found in the literature to model the innate HIS. Essentially, two distinct approaches 
are used: ordinary differential equations (ODEs) and partial differential equations (PDEs). 

\subsection*{Models based on ODEs}

The authors of \cite{Kumar2004} presented a model of inflammation that is based on ODEs and considers three types of cells/molecules: 
the pathogen and two inflammatory mediators. This model was able to reproduce some experimental results depending on the values used 
for initial conditions and parameters. The authors described the results of the sensitivity analysis, which suggests some therapeutic 
strategies. Their work was then extended\cite{reynolds1} to investigate the influence of time on an anti-inflammatory response. The 
mathematical model presented in \cite{reynolds1} consists of a system of ODEs with four equations that model: a) the pathogen; b) the 
active phagocytes; c) tissue damage; and d) anti-inflammatory mediators. The source term of the phagocytes, in other words, a term 
that models the entry of new phagocytes into the infected tissue, is a function that depends on a) the concentration of phagocytes; 
b) the concentration of pathogens; and c) tissue damage. This term models the different interactions that phagocytes can undergo 
during an immune response, whether the interactions are direct or mediated by cytokines. In the interaction mediated by cytokines, 
they consider only the implicit presence of cytokines. For example, in an immune response, the interaction of phagocytes with tissue 
is mediated by pro-inflammatory cytokines produced by infected epithelial tissue cells, and this relationship is modeled directly in 
the source term of the phagocytes. This representation contrasts with the model proposed in the current work, where cytokines and 
all their interactions are explicitly represented. 

A new adaptation of the first model \cite{Kumar2004} was proposed to simulate many scenarios involving repeated doses of 
endotoxin \cite{reynolds2}. This work applied results obtained through experiments using mice to guide \textit{in silico} experiments 
seeking to reproduce these results qualitatively. The mathematical model represents the key aspects of an acute inflammatory response, 
specifically when repeated doses of endotoxin are administered. This model replaces the pathogen equation proposed in the authors' 
previous work \cite{reynolds1} with an equation incorporating the endotoxin. In their simulations, they observed that the timing and 
magnitude of endotoxin doses, as well as the dynamics between pro- and anti-inflammatory mediators, are key to distinguishing between 
potentiation and tolerance phenomena \cite{reynolds2}. The authors also argued that their model, although simplified, nevertheless 
incorporates sufficiently complex dynamics to qualitatively reproduce a set of experimental results associated with different endotoxin 
administrations in mice. 

One final work \cite{Vodovotz2006} developed a more complete system of ODEs that models acute inflammation. This model includes 
macrophages, neutrophils, dendritic cells, TH1 cells, blood pressure, tissue trauma, effector elements such as iNOS, 
$\mathrm{NO_2^-}$ and $\mathrm{NO_3^-}$, pro-inflammatory and anti-inflammatory cytokines, and coagulation factors. In this model, 
as well as our own (described in detail in the next section), 
neutrophils and macrophages are directly activated by LPS. Moreover, activation also occurs indirectly by way of various stimuli 
consistently elicited after a trauma or hemorrhage. However, the model proposed by \cite{Vodovotz2006} does not explicitly include 
initial events of inflammation such as mast cell degranulation and complement activation, although these 
factors were incorporated implicitly into cytokine and endotoxin dynamics. The model also includes anti-inflammatory cytokines such 
as IL-10 and TGF $\beta$, in addition to soluble receptors for pro-inflammatory cytokines. The authors argued that their model proved 
useful in simulating the inflammatory response induced in mice by endotoxin, trauma and surgery or surgical bleeding, as it can predict 
levels of TNF, IL-10, IL-6 and reactive products of NO (NO$_{2}^{-}$ and NO$_{3}^{-}$) to some extent. 

\subsection*{Models based on PDEs}

The model proposed by Su \textit{et al} \cite{localmodel} uses a system of PDEs to represent the spatial dynamics of the innate and 
adaptive immune systems. It considers the simplest form of antigens, the molecular constituents of pathogen patterns, taking into 
account all the basic factors of an immune response: antigens, cells of the immune system, cytokines and chemokines. This model 
captures the following stages of immune response: recognition, initiation, effector response and resolution of infection or change 
to a new steady state. Accordingly, it can reproduce important phenomena such as a) the temporal order of cell arrival at the site 
of infection; b) antigen presentation by dendritic cells, macrophages and the involvement of regulatory T cells (Treg) in the
resolution of the immune response; c) the production of pro-inflammatory and anti-inflammatory cytokines; and d) chemotaxis. This 
model has formed the basis for the development of our work.

\section*{Mathematical model}\label{mathmodel}

The complete modeling of the HIS demands that a huge amount of work be performed by a large multidisciplinary team. In this work, 
we focus on a specific task: the development of a mathematical model of the innate immune response to the injection of LPS in a 
section of tissue, as well as such a model's computational implementation. One motivation for developing a model of the innate 
immune system is the fact that few such models are available in the literature; the majority of available models solely focus on 
the adaptive immune system. Another reason in favor of modeling the innate immune system is that many diseases result from the 
malfunction of the innate immune system; for these diseases, our proposed model could contribute to the definition of therapeutic 
strategies. In addition, a better comprehension of the inner workings of the separate parts composing the innate immune system is 
fundamental to a better understanding of immune response as a whole, as the innate immune system is responsible for both initiating 
the immune response and triggering the adaptive immune system. 

Our objective is to develop a parameterized mathematical model of the human innate immune system that simulates the immune response 
occurring in a generic tissue. To achieve this goal, we first build a model of the immune response to LPS. We have chosen to use
LPS because it is the major component of the outer membrane of Gram-negative bacteria, acting as an endotoxin substance that elicits 
strong immune responses; thus, it represents a vast number of inflammatory diseases. However, our proposed model is generic in the 
sense that it can be easily adapted to specific pathogens and distinct types of tissue through the adjustment of its parameters.

The mathematical model simulates the temporal and spatial behavior of lipopolysaccharide ($LPS$), macrophages, neutrophils ($N$), 
apoptotic neutrophils ($ND$), pro-inflammatory cytokines ($CH$), anti-inflammatory cytokine ($AC$) and protein granules ($G$). 
Macrophages are present in two states of readiness: resting ($RM$) and \textit{hyperactivated} ($AM$). The different subsets of 
protein granules \cite{Niels1997} released by neutrophils during their extravasation from blood vessels to the tissues are represented 
by a unique variable. Additionally, we must stress that the equations modeling pro- and anti-inflammatory cytokines are generic in 
the sense that they model the role of distinct cytokines taking part in the inflammatory process. Equation parameters can be adjusted 
to model the role of a specific pro- or anti-inflammatory cytokine. 

Our model extends the model proposed by Su \textit{et al} \cite{localmodel} by considering a macroscopic or homogenized view of a 
tissue. In \cite{localmodel}, the exchange between the vascular system (arterioles and vessels) and tissue was assumed to occur 
only at the boundaries of the domain, via Dirichlet boundary conditions. Our model allows each point of the tissue to be irrigated 
by arterioles and vessels, so that cells in the blood stream can enter into the tissue at any point. This is equivalent to a 
two-domain model, in which one domain represents the concentration of immune cells in the vascular system (in our case, neutrophils, 
$Nmax(x,t)$, and macrophages, $Mmax(x,t)$) and the other domain represents the different cells and molecules present in the tissue 
(our model considers lipopolysaccharide ($LPS$), neutrophils ($N$), apoptotic neutrophils ($ND$), pro-inflammatory cytokines ($CH$), 
anti-inflammatory cytokines ($AC$), protein granules ($G$), resting ($RM$) and \textit{hyperactivated} ($AM$) macrophages).
Communication between the two different domains is possible and is modeled by permeabilities that vary in space and time and may 
depend on the concentration of different cells and molecules (in our model, the endothelium permeability of neutrophils depends on 
the concentration of $CH$, whereas the permeability to macrophages depends on the concentration of both $CH$ and $G$). Figure 
\ref{modelPicture} presents our two-domain macroscopic model.

%inicio xandao
%\begin{figure}[thpb]
%	 \centering
%	 \includegraphics[scale=0.6]{img/AlexandrePigozzo_fig1.jpg}
%            \caption{Schematic representation of the two-domain model. The extravasation of neutrophils (or macrophages) from blood 
%to tissue depends on local permeability, $P(x,t)$, and on the difference between local concentrations of neutrophils in the two domains, 
%$N_{max}(x,t) - N(x,t)$. The permeability of the endothelium, which separates the two domains, varies with regard to time and space and depends 
%on the local presence of pro-inflammatory cytokines and protein granules.}
%	  \label{modelPicture}
%\end{figure}
%fim xandao

The main characteristics of the proposed model are: 
\begin{itemize}
\item Macrophages and neutrophils cooperate to mount a more effective and intense response against the LPS; 
\item The endothelium's permeability may vary with time and space and also depends on the local concentration of pro-inflammatory 
cytokine and protein granules, as depicted by Figure \ref{modelPicture}; 
\item Active macrophages regulate immune responses through the production of anti-inflammatory cytokines and the phagocytosis of 
apoptotic neutrophils; 
\item Anti-inflammatory cytokines perform a key role in the control of the inflammatory response, avoiding a state of persistent 
inflammation after the complete elimination of LPS. 
\end{itemize}

Figure \ref{relacoesExtendido} depicts the relationships among all of the model's components. Neutrophils, resting macrophages and 
active macrophages phagocytose the LPS. The neutrophils then undergo apoptosis, which may or may not be induced by the phagocytosis 
process. In this different state, apoptotic neutrophils cannot perform phagocytosis or produce pro-inflammatory cytokines; as a result, 
apoptotic neutrophils are eliminated from the body after being phagocytosed by active macrophages. The number of apoptotic neutrophils 
in the serum is an indirect indication of the probability that the immune response will cause tissue damage, because apoptotic 
neutrophils will die after a period of time, releasing cytotoxic granules and degradation enzymes in the medium that can cause tissue 
damage. Neutrophils produce pro-inflammatory cytokines, such as TNF-$\alpha$ and IL-8, as well as protein granules, which allow the 
direct activation and adhesion of monocytes in the endothelium of blood vessels, facilitating monocytes' extravasation into the tissue. 
The resting macrophages become active when they recognize the LPS. The pro-inflammatory cytokines produced by neutrophils and active 
macrophages increase the permeability of the blood vessels; consequently, more neutrophils and monocytes are recruited to the tissue. 
In addition, the pro-inflammatory cytokines act as a chemoattractant substance to the resting macrophages, active macrophages and 
neutrophils. The production of the pro-inflammatory cytokine is blocked when an active macrophage or neutrophil comes in contact with 
an anti-inflammatory cytokine. Macrophage activation is also blocked by the action of an anti-inflammatory cytokine, which is produced 
by active macrophages and by resting macrophages that are in contact with apoptotic neutrophils. 

%\begin{figure}[thpb]
%	 \centering
%	 \includegraphics[scale=0.45]{img/AlexandrePigozzo_fig2.png}
%           \caption{Relations among the components of the model.}
%	  \label{relacoesExtendido}
%\end{figure}


Below, we provide the equations derived from the model. Equation \ref{eqLPS} provides the LPS differential equation.

\begin{equation}
 \begin{cases} 
  RM_{activation} = \frac{\phi_{RM|LPS}.RM.LPS}{(1 + \theta_{AC}.AC)}\\ \\
  \frac{\partial LPS}{\partial t} = -\mu_{LPS} LPS - RM_{activation} -(\lambda_{N|LPS}N + \lambda_{AM|LPS}AM).LPS + D_{LPS}\Delta LPS  \\\\
  LPS(x,0) = LPS_0,\frac{\partial LPS(.,t)}{\partial n} |_{\partial\Omega} = 0 
\end{cases} 
\label{eqLPS}
\end{equation}

In this equation, $\mu_{LPS} LPS$ denotes the decay of LPS, where $\mu_{LPS}$ is the rate of decay, $RM_{activation}$ denotes 
the activation of resting macrophages, where $\phi_{RM|LPS}$ is the rate of activation. This activation occurs when resting macrophages 
recognize the LPS, after which macrophages phagocytose the LPS. $\lambda_{N|LPS}.N$ denotes the phagocytosis of LPS by neutrophils, 
where $\lambda_{N|LPS}$ is the rate of this phagocytosis. $\lambda_{AM|LPS}.AM$ denotes the phagocytosis of LPS by active macrophages, 
where  $\lambda_{AM|LPS}$ is the rate of this phagocytosis. $D_{LPS}\Delta LPS$ denotes LPS diffusion, whereas $D_{LPS}$ represents the 
diffusion coefficient. Figure \ref{diffusion} presents a schematic representation of the diffusion process implemented by the diffusion 
operator, $D_{LPS}\Delta LPS$, and illustrates the diffusion of cells through the tissue. Diffusion is defined as the spread of particles 
from regions of higher concentration to regions of lower concentration. 

%\begin{figure}[thpb]
%	 \centering
%	 \includegraphics[scale=0.55]{img/AlexandrePigozzo_fig3.jpg}
%            \caption{Schematic representation of the diffusion process.}
%	  \label{diffusion}
%\end{figure}

The differential equation corresponding to the resting macrophage (RM) is given in Equation \ref{eqMREstendido}.

\begin{equation}
\label{eqMREstendido}
 \begin{cases} 
RM_{P} = (P^{max}_{RM} - P^{min}_{RM}).\frac{CH}{(CH + keqch)} + P^{min}_{RM}\\\\
RM_{Q} = (Q^{max}_{RM} - Q^{min}_{RM}).\frac{G}{(G + keq\_g)} + Q^{min}_{RM}\\\\
source_{RM} = (RM_{P} + RM_{Q}).(M^{max} - (RM + AM))\\  \\
\frac{\partial RM}{\partial t} = -\mu_{RM} RM - RM_{activation} + D_{RM} \Delta RM + source_{RM} - \nabla. (\chi_{RM} RM \nabla CH)  \\\\
RM(x,0) = RM_0, \frac{\partial RM(.,t)}{\partial n} |_{\partial\Omega} = 0
 \end{cases}
\end{equation}

$RM_{P}$ and $RM_{Q}$ denote the increase in endothelium permeability and its effects on monocyte extravasation. The permeability of 
blood vessel endothelium is modeled by a Hill equation \cite{Goutelle2008}, which also has been used to model drug dose-response 
relationships \cite{Wagner1968}. The idea is to model the increase in the permeability of the endothelium in accordance with the 
number of pro-inflammatory cytokines deposited on the endothelium. Figure \ref{permeability} illustrates the effect of increasing  
blood vessel permeability. We can see that the space between two neighboring endothelial cells increases, 
allowing more cells to extravasate to the tissue. The dynamic permeability depends on the cytokine concentration. 

%\begin{figure}[thpb]
%	 \centering
%	 \includegraphics[scale=0.6]{img/AlexandrePigozzo_fig4.jpg}
%           \caption{Representation of the differences between fixed and dynamic permeabilities.}
%	  \label{permeability}
%\end{figure}

The calculation of $RM_{P}$ involves the following parameters: a) $P^{max}_{RM}$, the maximum endothelium permeability induced by the 
pro-inflammatory cytokine; b) $P^{min}_{RM}$, the minimum endothelium permeability induced by the pro-inflammatory cytokine; and 
c) $keqch$, the number of pro-inflammatory cytokines that exert $50\%$ of the maximum effect on permeability. 

$RM_{Q}$ denotes the increase in endothelium permeability induced by protein granules, and its calculation is similar to that of $RM_{P}$, 
except for the parameters involved: $Q^{max}_{RM}$, $Q^{min}_{RM}$ and $keq\_g$. $source_{RM}$ denotes the source term of macrophages, 
which is related to the number of monocytes that will enter into the tissue from the blood vessels. 
This number depends on the endothelium permeability $RM_{P} + RM_{Q}$ and on the number of monocytes appearing in the blood ($M^{max}$).
 
$\mu_{RM} RM$ denotes resting macrophage apoptosis, where $\mu_{RM}$ is the apoptosis rate. $RM_{activation}$, as explained above, models 
the activation of resting macrophages and denotes the number of resting macrophages that are becoming active. The term $D_{RM} \Delta RM$ 
denotes the resting macrophage diffusion, where $D_{RM}$ is the diffusion coefficient. $\nabla. (\chi_{RM} RM \nabla CH)$ denotes the 
resting macrophage chemotaxis, where $\chi_{RM}$ is the chemotaxis rate.

Figure \ref{chemotaxis} provides a schematic representation of the chemotaxis process implemented by the chemotaxis operator,  
$\nabla. (\chi_{RM} RM \nabla CH)$. Chemotaxis is the phenomenon by which cells direct their own movements according to certain chemicals 
present in their environment.

%\begin{figure}[thpb]
%	 \centering
%	 \includegraphics[scale=0.55]{img/AlexandrePigozzo_fig5.jpg}
%            \caption{Schematic representation of chemotaxis.}
%	  \label{chemotaxis}
%\end{figure}

The differential equation corresponding to the active macrophage (AM) is given in Equation \ref{eqMAEstendido}.

\begin{equation}
\label{eqMAEstendido}
 \begin{cases} 
\frac{\partial AM}{\partial t} = -\mu_{AM} AM + RM_{activation} + D_{AM} \Delta AM - \nabla. (\chi_{AM} AM \nabla CH)\\\\
AM(x,0) = AM_0, \frac{\partial AM(.,t)}{\partial n} |_{\partial\Omega} = 0 
\end{cases}
\end{equation}


Above, $\mu_{AM} AM$, $D_{AM} \Delta AM$, and $\nabla. (\chi_{AM} AM \nabla CH)$ denote the active macrophage apoptosis, diffusion, 
and chemotaxis, respectively, whereas $\mu_{AM}$, $D_{AM}$, and $\chi_{AM}$ are the apoptosis rate, diffusion coefficient, and chemotaxis 
rate, respectively.

The differential equation for the pro-inflammatory cytokine (CH) is given in Equation \ref{eqCHEstendido}.

\begin{equation}
\label{eqCHEstendido}
\begin{cases} 
\frac{\partial CH}{\partial t} = -\mu_{CH} CH + ((\beta _{CH|N}.N.LPS + \beta _{CH|AM}.AM.LPS).(1 - \frac{CH}{chInf}))/(1 + \theta_{AC}.AC)+ \\
 + D_{CH} \Delta CH \\ \\
CH(x,0) =  CH_0, \frac{\partial CH(.,t)}{\partial n} |_{\partial\Omega} = 0 
\end{cases}
\end{equation}

In this equation, $\mu_{CH} CH$ denotes the pro-inflammatory cytokine decay, where $\mu_{CH}$ is the decay rate. 
$\beta _{CH|N}.N$ denotes the pro-inflammatory cytokine production by the neutrophils, where $\beta _{CH|N}$ is the production rate. 
$\beta _{CH|AM}.AM$ denotes the pro-inflammatory cytokine production by active macrophages, where $\beta _{CH|AM}$ is the production rate. 
The saturation of cytokine production by active macrophages is calculated by the equation $(1 - \frac{CH}{chInf})$, where $chInf$ is an 
estimate of the maximum quantity of pro-inflammatory cytokine supported by the tissue. 
The production of pro-inflammatory cytokine decreases when anti-inflammatory cytokine acts on the producing cells. 
This influence of anti-inflammatory cytokine is denoted by the expression $1/(1 + \theta_{AC}.AC)$. 
$D_{CH} \Delta CH$ models pro-inflammatory cytokine diffusion, where $D_{CH}$ is the diffusion coefficient. 

The neutrophil differential equation (N) is given in Equation \ref{eqNeutrofilo}.

\begin{equation}
 \begin{cases}   
P_{N} = (P^{max}_{N}- P^{min}_{N}).\frac{CH}{CH + Keqch} + P^{min}_{N}\\\\
source_{N} = P_{N}.(N^{max} - N)\\\\
\frac{\partial N}{\partial t} = -\mu_N N -\lambda _{LPS|N} LPS. N + D_N \Delta N + source_{N} - \nabla. (\chi_N N \nabla CH) \\\\
N(x,0) = N_0, \frac{\partial N(.,t)}{\partial n} |_{\partial\Omega} = 0 
\end{cases} 
\label{eqNeutrofilo}
\end{equation}

In this equation, $P_{N}$ denotes the increase in endothelium permeability and its effects on neutrophil extravasation. 
In the top equation, $P^{max}_{N}$ is the maximum endothelium permeability induced by pro-inflammatory cytokines, 
$P^{min}_{N}$ is the minimum endothelium permeability induced by pro-inflammatory cytokines and $keqch$ is the number of pro-inflammatory 
cytokines that exert $50\%$ of the maximum effect on endothelium permeability. 

Here, $\mu_N N$ denotes neutrophil apoptosis, where $\mu_N$ is the rate of apoptosis. $\lambda _{LPS|N} LPS. N$ denotes the neutrophil 
apoptosis induced by phagocytosis, where $\lambda _{LPS|N}$ represents the rate of this induced apoptosis. The term $D_N \Delta N$ denotes
neutrophil diffusion, where $D_N$ is the diffusion coefficient. $source_{N}$ represents the source term of neutrophil, i.e., the number of 
neutrophils entering the tissue from the blood vessels. This number depends on the endothelium permeability ($P_{N}$) and on the number of
neutrophils in the blood ($N^{max}$). The term $\nabla. (\chi_N N \nabla CH)$ denotes the chemotaxis process of the neutrophils, 
where $\chi_N$ represents the chemotaxis rate. 

The differential equation corresponding to the apoptotic neutrophil (ND) is given in Equation \ref{eqNDEstendido}.

\begin{equation}
\label{eqNDEstendido}
\begin{cases} 
\frac{\partial ND}{\partial t} = \mu_N N + \lambda_{LPS|N} LPS. N - \lambda_{ND|AM} ND.AM + D_{ND} \Delta ND  \\\\
ND(x,0) = ND_0, \frac{\partial ND(.,t)}{\partial n} |_{\partial\Omega} = 0 
\end{cases}
\end{equation}

Here, note that $\mu_N N$ and $\lambda _{LPS|N} LPS. N$ were defined previously, whereas $\lambda _{ND|AM} ND.AM$ denotes the 
phagocytosis of the apoptotic neutrophil carried out by active macrophages, and $\lambda _{ND|AM}$ is the rate of this phagocytosis. 
$D_{ND} \Delta ND$ models the apoptotic neutrophil diffusion, where $D_{ND}$ is the diffusion coefficient. 

The differential equation for protein granules (G) is given in Equation \ref{eqGEstendido}.

\begin{equation}
\label{eqGEstendido}
\begin{cases} 
\frac{\partial G}{\partial t} = -\mu_G G + \alpha _{G|N}.source_{N}.(1 - \frac{G}{gInf}) + D_{G} \Delta G \\\\
G(x,0) =  G_0, \frac{\partial G(.,t)}{\partial n} |_{\partial\Omega} = 0 
\end{cases}
\end{equation}

$\mu_G G$ models the decay of the granules, where $\mu_G$ is the decay rate. $\alpha _{G|N}.source_{N}$ denotes the production of 
protein granules by neutrophils extravasating from the blood into inflamed tissue, where $\alpha_{G|N}$ is a dimensionless constant. 
The saturation of protein granule production is calculated by the expression $(1 - \frac{G}{gInf})$, where $gInf$ is the maximum number 
of protein granules. $D_{G} \Delta G$ models protein granule diffusion, where $D_{G}$ is the diffusion coefficient.

The differential equation for the anti-inflammatory cytokine (AC) is given in Equation \ref{eqCAEstendido}.

\begin{equation}
\label{eqCAEstendido}
\begin{cases} 
\frac{\partial AC}{\partial t} = -\mu_{AC} AC + (\beta_{RM|ND}.RM.ND + \alpha_{AC|AM}.AM).(1 - \frac{AC}{acInf}) + D_{AC} \Delta AC \\\\
AC(x,0) =  AC_0, \frac{\partial AC(.,t)}{\partial n} |_{\partial\Omega} = 0 
\end{cases}
\end{equation}

In this equation, $\mu_{AC} AC$ denotes the anti-inflammatory cytokine decay, where $\mu_{AC}$ represents the decay rate. 
$\beta _{RM|ND}.RM.ND$ denotes the anti-inflammatory cytokine production by the resting macrophages in the presence of apoptotic 
neutrophils, where $\beta _{RM|ND}$ is the rate of this production. 
$\alpha _{AC|AM}.AM$ denotes the anti-inflammatory cytokine production by active macrophages, where this production has 
rate $\alpha _{AC|AM}$ and saturation $(1 - \frac{AC}{acInf})$, where $acInf$ is the maximum number of anti-inflammatory cytokines 
in the tissue. $D_{AC} \Delta AC$ models the anti-inflammatory cytokine diffusion, where $D_{AC}$ is the diffusion coefficient.


\section*{Implementation}\label{implementation}

The numerical method that we have applied to our mathematical model is presented in our previous work \cite{icaris11}.

% a method commonly used in the numeric discretization of PDEs. The FDM is a method of resolution of differential 
% equations and is based on the approximation of derivatives using finite differences. 
% 
% Below, we provide an example of a finite difference operator used in the discretization of the Laplace operator, which simulates 
% the diffusion phenomenon:
% 
% \begin{equation}
%  \label{discretizacaoDifusao}
%  D_N \frac{\partial^2 N(x)}{\partial x^2} \approx D_N*((n[x+deltaX] -2*n[x] +n[x-deltaX])/deltaX^2))
% \end{equation}

% In Equation \ref{discretizacaoDifusao}, $n$ represents the discretization of the neutrophil population, $D_N$ is the diffusion 
% coefficient of neutrophils, $x$ is the position in the space and $deltaX$ is the space discretization. A complex part of PDE 
% resolution involves the convective term (the chemotaxis term). The development of numerical methods to approximate 
% convective terms (which in most cases are nonlinear) has been the subject of intense research in the past few 
% years \cite{Harten1997, Leonard1988, Shu1989, Sod1978}. Different numerical approaches have been proposed for the discretization 
% of the chemotaxis term \cite{chemotaxismath1,chemotaxismath2}. Our implementation is based on the FDM for spatial discretization 
% and the explicit Euler method for time evolution. The discretization of the chemotaxis term ($\nabla. (\chi_N N \nabla CH)$) uses 
% the First-Order Upwind scheme \cite{Hafez2002}. Therefore, the precision of our numerical implementation is first order with respect 
% to both time (explicit Euler) and space (upwind scheme). The upwind scheme discretizes hyperbolic PDEs through the use of differences 
% with bias in the direction indicated by the signal of characteristic speeds. This method uses an adaptive or solution-sensitive stencil 
% to numerically and more precisely simulate the direction of information propagation. 
% 
% In one dimension, the upwind scheme approximates the chemotaxis term as the sum of $flux\_left$ at the point $x - \frac{deltaX}{2}$ 
% and $flux\_right$ at the point $x + \frac{deltaX}{2}$ in the following way:
% \newline
% 
% \begin{center}
% \begin{lstlisting}
% if (ch[x] - ch[x-deltaX] > 0) {
%     flux_left = -(ch[x] - ch[x-deltaX])*n[x-deltaX]/deltaX;
% }
% else{
%     flux_left = -(ch[x] - ch[x-deltaX])*n[x]/deltaX;
% }
% 
% 
% if (ch[x+deltaX] - ch[x] > 0) {
%     flux_right = (ch[x+deltaX] - ch[x])*n[x]/deltaX;
% }
% else{
%     flux_right = (ch[x+deltaX] - ch[x])*n[x+deltaX])/deltaX;
% }
% 
% \end{lstlisting}
% \end{center}
% 
% In this code, $ch$ represents the discretization of pro-inflammatory cytokine, $n$ denotes the discretization of neutrophils, 
% $x$ is the position in space and $deltaX$ is the spatial discretization. The test is to define the signal of the characteristic 
% speed, where the speed of $N(x)$ is given by $\nabla CH$. This value is then used to choose between the forward and backward 
% schemes of finite differences.
% 
% We decided to implement our own numerical method to solve the system of PDEs because we saw the opportunity to parallelize the code 
% and because most numerical libraries offer few functions suitable to our problem. The sequential code was implemented in C. 

We executed some convergence tests to test the implementation of our numerical method. 
In short, we assumed that the correct solution derived from the results of a very refined mesh, where the refinement was in terms of 
time ($dt = 10^{-6} day$) and space ($deltaX = 0.1 mm$). To show convergence with respect to time, we selected two new values for 
$dt$, $dt1 = 4.0 \times 10^{-6} day$ and $dt2 = 8.0 \times 10^{-6} day$. We applied the L2-norm to compute the errors when using 
$dt1$ and $dt2$ for our refined mesh. We observed that the error when using $dt2$ was $2.3$ times greater than the error obtained 
with $dt1$. Therefore, as theoretically predicted, our numerical scheme is first-order accurate with respect to time. We then conducted 
the same analysis for convergence with respect to space, choosing two new values of $deltaX$, $dx1 = 0.4 mm$ and $dx2 = 0.8 mm$. 
The L2-norm error when using $dx2$ was $2.03$ times greater than the error obtained with $dx1$. Once again, the values obtained were 
as expected, as we were using a first-order discretization (upwind) in space. These results gave us confidence that our numerical 
solver had been correctly implemented.

% We also parallelized the code by applying three distinct models: shared memory, message passing, and a hybrid model. More details about 
% the parallel code can be obtained in \cite{Pigozzo2010}.

\section*{Numerical experiments}\label{results}

The model's initial conditions and parameters are given in Tables \ref{initialConditions} and \ref{parameters}, respectively. 
In our simulations, we assumed a one-dimensional domain of 5 $mm$ length and a simulation time of 5 days. 
In fact, this one-dimensional model is a simplification of a 3D block model in that we have assumed that the lengths associated 
with $y$ and $z$ are much smaller than the length associated with $x$.

In this paper, we obtained parameter values for humans whenever they were available. 
%inicio xandao
 We chose values for the initial concentrations of LPS according to the work of the authors in \cite{Movat1986}. 
In their experiments, \textit{E. coli} cells were inoculated intradermally ($10^8$) into normal and neutropenic rabbits. 
They reported that all bacteria and inflammatory cells were contained in this 1.5 cm diameter biopsy and restricted to its 0.2 cm 
thick layer of dermal collagen. Thus, the volume of dermis in which the \textit{E. coli} cells were contained was approximately 
$0.35$ cm$^3$ \cite{Li2004}. This finding suggested us a value of $LPS_0 = 100.0 \times 10^4$ cells$/mm^3$. 

%fim_xandao 

In Table \ref{parameters}, parameters marked with * were adjusted to qualitatively reproduce the results obtained in several 
studies of the immune response to LPS. In the case of LPS, we adjust the equation parameters to obtain an exponential decrease, 
as shown in \cite{Li2002}. The results of the concentration of pro-inflammatory cytokines over time are qualitatively similar to 
those obtained in some experimental works \cite{Malefyt01111991, Oswald1992, 2007851}. The time course for the anti-inflammatory 
cytokine is qualitatively similar to the results in \cite{Malefyt01111991}. An important feature present in our model is the 
inhibition of the production of pro-inflammatory cytokines by neutrophils through the action of anti-inflammatory 
cytokines \cite{Marie1996}. The protein granule model behavior is based on existing work \cite{19652869}.
The parameters marked with ** were based on the values given in the references but were adjusted due to the use of distinct units 
(for example, from L to $mm^3$) or to fit in a 5 $mm$ tissue. 


%fim xandao

\begin{table}[htpb]
\centering
\caption{Initial conditions}
\begin{tabular}{cccc}
\hline
\textbf{Parameter} & \textbf{Value} & \textbf{Unit}   \\
\hline 
$LPS_0$ & $100: \hspace{3 mm}  0 < x < 1$ & $10^4 cells/mm^3$ \\
\hline
$LPS_0$ & $0: \hspace{3 mm}  1 \le x < 5$ & $10^4 cells/mm^3$ \\
\hline
$RM_0$ & $1: \hspace{3 mm} 0 < x < 5$ & $10^4 cells/mm^3$ \\
\hline 
$AM_0$ & $0: \hspace{3 mm} 0 < x < 5$ & $10^4 cells/mm^3$ \\
\hline 
$CH_0$ & $0: \hspace{3 mm} 0 < x < 5$ & $10^4 cells/mm^3$ \\
\hline 
$N_0$ & $0: \hspace{3 mm} 0 < x < 5$ & $10^4 cells/mm^3$ \\
\hline 
$ND_0$ & $0:\hspace{3 mm} 0 < x < 5$ & $10^4 cells/mm^3$ \\
\hline  
$G_0$ & $0:\hspace{3 mm} 0 < x < 5$ & $10^4 cells/mm^3$ \\
\hline 
$AC_0$ & $0:\hspace{3 mm} 0 < x < 5$ &  $10^4 cells/mm^3$ 
\end{tabular}
\label{initialConditions} 
\end{table}


\begin{table}[htpb]
\centering
\caption{Parameters}
\begin{tabular}{cccc}
\hline
\textbf{Parameter} & \textbf{Value} & \textbf{Unit}  & \textbf{Reference} \\
\hline 
$\phi_{RM|LPS}$ & 0.1 &  1/($cells/mm^3$ ).day & \cite{localmodel}**\\
\hline
$\theta_{AC}$ & 1 &  1/($cells/mm^3$ ) & estimated*\\
\hline 
$\mu_{LPS}$ & 0.005 & 1/day & \cite{localmodel}\\
\hline
$\lambda_{N|LPS}$ & 0.55 & 1/($cells/mm^3$ ).day & \cite{localmodel}\\
\hline 
$\lambda_{AM|LPS}$ & 0.8 & 1/($cells/mm^3$ ).day &\cite{localmodel} \\
\hline
$D_{LPS}$ & 2000 & ${\mu m}^2$/day & estimated*\\ 
\hline
$P^{max}_{RM}$ & 0.1 & 1/day & estimated*\\
\hline 
$P^{min}_{RM}$ & 0.01 & 1/day & estimated*\\
\hline 
$Q^{max}_{RM}$ & 0.5 & 1/day & estimated*\\
\hline 
$Q^{min}_{RM}$ & 0 & 1/day & estimated*\\
\hline 
$keqch$ & 1 &  $cells/mm^3$ & estimated*\\
\hline 
$keq_g$ & 1 &  $cells/mm^3$ & estimated*\\
\hline 
$M^{max}$ & 6 &  $cells/mm^3$ & estimated*\\
\hline 
$\mu_{RM}$ & 0.033 & 1/day & \cite{localmodel}\\
\hline 
$D_{RM}$ & 4320 & ${\mu m}^2$/day & \cite{Gammack2004,Owen97} \\
\hline 
$\chi_{RM}$ & 3600 & ${\mu m}^2$/day & \cite{Lauffenburger1993,Sozzani1991,Tranquillo1990} \\
\hline 
$\mu_{AM}$ & 0.07 & 1/day & \cite{localmodel}\\
\hline 
$D_{AM}$ & 3000 & ${\mu m}^2$/day & \cite{Gammack2004,Owen97} \\
\hline 
$\chi_{AM}$ & 4320 & ${\mu m}^2$/day & \cite{Lauffenburger1993,Sozzani1991,Tranquillo1990} \\
\hline
$\mu_{CH}$ & 7 & 1/day & \cite{localmodel}**\\
\hline 
$\beta_{CH|N}$ & 1 & 1/($cells/mm^3$).day & \cite{Andoh2000}\\
\hline 
$\beta_{CH|AM}$ & 0.8 & 1/($cells/mm^3$).day & \cite{Andoh2000}\\
\hline 
$chInf$ & 3.6 & $cells/mm^3$ & \cite{Waal1991}** \\
\hline 
$D_{CH}$ & 9216 &${\mu m}^2$/day & \cite{localmodel,Gammack2004}\\
\hline
$P^{max}_{N}$ & 11.4 & 1/day & \cite{Price1994}**\\
\hline 
$P^{min}_{N}$ & 0.0001 & 1/day & estimated*\\
\hline 
$keqch$ & 1 &  $cells/mm^3$ & estimated*\\
\hline 
$N^{max}$ & 8 &  $cells/mm^3$ & estimated*\\
\hline 
$\mu_N$ & 3.43 & 1/day & \cite{Edwards2003}\\
\hline 
$\lambda_{LPS|N}$ & 0.55 & 1/($cells/mm^3$).day & \cite{localmodel}\\
\hline 
$D_N$ & 12096 & ${\mu m}^2$/day & \cite{Felder1994} \\
\hline 
$\chi_{N}$ & 14400 & ${\mu m}^2$/day & \cite{Chettibi1994}\\
\hline 
$\lambda _{ND|AM}$ & 2.6 & 1/($cells/mm^3$).day & \cite{localmodel}\\
\hline
$D_{ND}$ & 0.144 & ${\mu m}^2$/day & \cite{localmodel}** \\
\hline 
$\mu_{G}$ & 5 & 1/day & estimated*\\
\hline 
$\alpha _{G|N}$ & 0.6 & dimensionless & estimated*\\
\hline 
$gInf$ & 3.1 & $cells/mm^3$ & estimated* \\
\hline 
$D_{G}$ & 9216 &${\mu m}^2$/day & estimated*\\
\hline 
$\mu_{AC}$ & 4 & 1/day & estimated*\\
\hline 
$\beta_{RM|ND}$ & 1.5 & 1/($cells/mm^3$).day & estimated*\\
\hline 
$\alpha_{AC|AM}$ & 1.5 & dimensionless & estimated*\\
\hline 
$acInf$ & 3.6 & $cells/mm^3$ & \cite{Waal1991}**\\
\hline 
$D_{AC}$ & 9216 &${\mu m}^2$/day & \cite{Gammack2004}
\end{tabular}
\label{parameters} 
\end{table}

In Figure \ref{figure2}, we initially inject LPS only into a small part of the tissue. 
As time progresses, we can see two important phenomena occurring: the diffusion of LPS through the tissue and the decrease of 
LPS mainly due to the action of neutrophils and macrophages.

%\begin{figure}[!htb]
%	 \centering
%	 \includegraphics[scale=0.65]{img/AlexandrePigozzo_fig6.png}	
%	 \caption{Temporal evolution of the spatial distribution of LPS.}
%	 \label{figure2}
%\end{figure}

In the case of neutrophils (Figure \ref{figure3}), we can witness an increase in neutrophil population mainly in regions of tissue 
having higher levels of LPS. This increase happens because of an increase in endothelium permeability in addition to the chemotaxis 
process attracting neutrophils to regions possessing more pro-inflammatory cytokines. When the amount of LPS is low, the neutrophil 
population stops growing and starts to decrease because fewer neutrophils are entering into the tissue.  

%\begin{figure}[!htb]
%	 \centering
%	 \includegraphics[scale=0.65]{img/AlexandrePigozzo_fig7.png}	
%	 \caption{Temporal evolution of the spatial distribution of neutrophils.}
%	 \label{figure3}
%\end{figure}

In Figure \ref{figure4}, we observe an increase of pro-inflammatory cytokines until 6 hours, when a large number of neutrophils are 
present in the tissue. Afterwards, the number of pro-inflammatory cytokines decreases, mainly due to the presence of a large number 
of active macrophages. Consequently, the anti-inflammatory cytokine population increases. The decrease of $CH$ has many important 
consequences: fewer neutrophils and monocytes are migrating to the inflamed tissue, and fewer macrophages are becoming active, as 
can be observed in Figure \ref{figure5}. This figure shows that the active macrophage population grows until 12 hours and then starts 
to decrease because, as explained before, anti-inflammatory cytokines inhibit the activation of resting macrophages.

%\begin{figure}[!htb]
%	 \centering
%	 \includegraphics[scale=0.65]{img/AlexandrePigozzo_fig8.png}	
%	 \caption{Temporal evolution of the spatial distribution of pro-inflammatory cytokine.}
%	 \label{figure4}
%\end{figure}

%\begin{figure}[!htb]
%	 \centering
%	 \includegraphics[scale=0.65]{img/AlexandrePigozzo_fig9.png}	
%	 \caption{Temporal evolution of the spatial distribution of  active macrophage.}
%	 \label{figure5}
%\end{figure}


\subsection*{Comparison of different scenarios}

To show the importance of some cells, molecules and processes in the dynamics of the innate immune response, 
we performed a set of simulations under different scenarios. Each simulation begins with a simple scenario in which we assume that 
only macrophages participate in the immune response to LPS (Case 1). We then consider progressively more complex scenarios. 
In each subsequent scenario, a new set of equations and terms are added to the previous one until the complete scenario is obtained 
(Case 5). 

A description of each case is given below: 
\begin{itemize}
\item Case 1: only macrophages participate in the immune response. Resting tissue-resident macrophages are responsible for the initial 
response to LPS. 
\item Case 2: considers a) the production of pro-inflammatory cytokines by active macrophages; and b) all effects of pro-inflammatory 
cytokines, such as the increase in permeability and chemotaxis. 
\item Case 3: incorporates neutrophils into the model, which participate in the immune response as a major phagocytic leukocyte. They 
are also responsible for producing pro-inflammatory cytokines. 
\item Case 4: incorporates protein granules into the model, which are produced by neutrophils and contribute to an increase in the 
endothelium's permeability, allowing more monocytes to enter into the tissues and differentiate in resting macrophages. 
\item Case 5: incorporates anti-inflammatory cytokines into the model. In this case, anti-inflammatory cytokines block the production 
of pro-inflammatory cytokines by the neutrophils and active macrophages. In addition, anti-inflammatory cytokines block the activation
of resting macrophages. 
\end{itemize}

Figure \ref{atempoEstendido} depicts the temporal evolution of the total amount of LPS in the tissue. 
Observe that the introduction of pro-inflammatory cytokines in Case 2 causes a small decrease in the amount of LPS when compared to Case 1. 
This decrease has occurred because our model considers the pro-inflammatory cytokine influence on monocyte migration to be almost 
negligible. 

In Case 3, the decrease in LPS has been accelerated due to the presence of neutrophils migrating into the tissue in huge quantities. 
The number of neutrophils in the tissue is enough to control the infection. 

In Case 4, observe that the extravasation of a second wave of monocytes (a consequence of the presence of protein granules produced by 
the neutrophils) has no impact on the potentiation of the immune response because the LPS has been almost completely eliminated. 
Note that the LPS decrease is smaller in case 5 than in cases 3 and 4. This fact is a consequence of the presence of anti-inflammatory 
cytokines in the model, which causes a decrease in the number of neutrophils and monocytes extravasating to the tissue. 

%\begin{figure}[!htb]
%	 \centering
%	 \includegraphics[scale=0.65]{img/AlexandrePigozzo_fig10.png}	
%	 \caption{Temporal evolution of the total quantity of LPS.}
%	 \label{atempoEstendido}
%\end{figure}

Figure \ref{mrtempoEstendido} depicts the temporal evolution of the population of resting macrophages and demonstrates 
that the introduction of pro-inflammatory cytokines, neutrophils and protein granules (Cases 2, 3 and 4) contributes to an increase in 
endothelium permeability, which in turn allows the entry of more monocytes. As a consequence, the number of resting macrophages increases, 
an increase compounded between Cases 4 and 5 because anti-inflammatory cytokines are blocking the activation of resting macrophages. 


%\begin{figure}[!htb]
%	 \centering
%	 \includegraphics[scale=0.65]{img/AlexandrePigozzo_fig11.png}	
%	 \caption{Temporal evolution of the resting macrophage population.}
%	 \label{mrtempoEstendido}
%\end{figure}

Figure \ref{matempoEstendido} presents the temporal evolution of the active macrophage population. Observe that an increase in this 
population occurs from Case 1 to Case 2, which is due to the production of pro-inflammatory cytokines by active macrophages, which 
have also increased permeability and chemotaxis. In Case 3, the introduction of neutrophils contributes to a faster elimination of
LPS, and as a result, less LPS is available to activate resting macrophages. When protein granules are included in the model (Case 4), 
we can observe an increase in the quantity of active macrophages. This population increase has occurred because protein granules allow 
the direct activation and adhesion of monocytes in the endothelium of blood vessels, thus facilitating the monocytes' extravasation to 
the tissues. Finally, a significant reduction in the total amount of active macrophages occurs in Case 5 due to the action of 
anti-inflammatory cytokines, which  block the activation of resting macrophages. In addition, anti-inflammatory cytokines block the 
production of pro-inflammatory cytokines, causing a decrease in endothelium permeability and consequently in the number of monocytes 
extravasating to the tissue.

%\begin{figure}[!htb]
%	 \centering
%	 \includegraphics[scale=0.65]{img/AlexandrePigozzo_fig12.png}	
%	 \caption{Temporal evolution of the active macrophage population.}
%	 \label{matempoEstendido}
%\end{figure}

Observe the significant increase in the number of pro-inflammatory cytokines in Figure \ref{chtempoEstendido} between Cases 2 and 3. 
This increase is a direct consequence of the incorporation of neutrophils into the model, as neutrophils produce a huge amount of 
pro-inflammatory cytokines. No change occurs between Cases 3 and 4 because the entry of more monocytes into the tissue occurs during 
termination of the immune response, when the LPS available to activate the monocytes is small. In Case 5, the reduction in production 
of pro-inflammatory cytokines due to the action of anti-inflammatory cytokines is responsible for the decrease in the total quantity 
of pro-inflammatory cytokines. 

%\begin{figure}[!htb]
%	 \centering
%	 \includegraphics[scale=0.65]{img/AlexandrePigozzo_fig13.png}	
%	 \caption{Temporal evolution of the pro-inflammatory cytokine population.}
%	 \label{chtempoEstendido}
%\end{figure}

Figure \ref{ntempoEstendido} depicts the temporal evolution of the neutrophil population, whose increase is similar in Cases 3 and 4 due 
to the fact previously stated: the entry of more monocytes into the tissue occurs during termination of the immune response. In Case 5, 
the number of neutrophils into the infected tissue is smaller than in Case 4 because fewer pro-inflammatory cytokines are present in the 
tissue, which results in a reduction in the number of neutrophils migrating into the infected tissue. 

%\begin{figure}[!htb]
%	 \centering
%	 \includegraphics[scale=0.65]{img/AlexandrePigozzo_fig14.png}	
%	 \caption{Temporal evolution of the neutrophil population.}
%	 \label{ntempoEstendido}
%\end{figure}

Figure \ref{ndtempoEstendido} illustrates a small decrease in the number of apoptotic neutrophils between Cases 3 and 4. This reduction 
is a consequence of the presence of more active macrophages in Case 4 than in Case 3. In Case 5, the presence of fewer active macrophages 
in the tissue leads to a reduction in the number of apoptotic neutrophils that are phagocytosed.

%\begin{figure}[!htb]
%	 \centering
%	 \includegraphics[scale=0.65]{img/AlexandrePigozzo_fig15.png}	
%	 \caption{Temporal evolution of the apoptotic neutrophil population.}
%	 \label{ndtempoEstendido}
%\end{figure}

Figure \ref{gtempoEstendido} shows an increase in the number of protein granules between Case 4 and Case 5. In Case 4, the number of 
neutrophils migrating to the infected tissue is larger, causing an increase in protein granule production as well. 

%\begin{figure}[!htb]
%	 \centering
%	 \includegraphics[scale=0.65]{img/AlexandrePigozzo_fig16.png}	
%	 \caption{Temporal evolution of the protein granule population.}
%	 \label{gtempoEstendido}
%\end{figure}

Finally, Figure \ref{catempoEstendido} depicts the temporal evolution of the anti-inflammatory cytokine population. Observe that the 
number of anti-inflammatory cytokines increases after the termination of the infection (as shown in Figure 10).

%\begin{figure}[!htb]
%	 \centering
%	 \includegraphics[scale=0.65]{img/AlexandrePigozzo_fig17.png}	
%	 \caption{Temporal evolution of the anti-inflammatory cytokine population.}
%	 \label{catempoEstendido}
%\end{figure}

\section*{Conclusions and future works}\label{conclusion}

In this work, we have presented a computational model for the dynamics of representative types of cells and molecules of the HIS 
during an innate response to the injection of LPS into a small section of tissue. To achieve this objective, we have proposed a 
mathematical model that incorporates the main interactions occurring between LPS and some cells and molecules 
of the innate immune system. The model proposes a macroscopic or homogenized view of tissue composed of two different domains: 
one domain represents the concentration of immune cells in the vascular system (in our case, neutrophils, $N_{max}(x,t)$, 
and macrophages, $M_{max}(x,t)$), whereas the other domain represents the different cells and molecules present in the tissue 
(our model considers lipopolysaccharide ($LPS$), neutrophils ($N$), apoptotic neutrophils ($ND$), pro-inflammatory cytokines ($CH$), 
anti-inflammatory cytokines ($AC$), proteins granules ($G$), resting ($RM$) and \textit{hyperactivated} ($AM$) macrophages). 
Communication between the two different domains is possible and is modeled by an endothelium permeability that varies in space and 
time and may depend also on the concentration of different cells and molecules (in our model, the endothelium's permeability to 
neutrophils depends on the concentration of $CH$, whereas its permeability to macrophages depends on $CH$ and $G$).  

The model proposed in this work has been able to reproduce several features present in immune responses, such as:

\begin{itemize}
\item the order of arrival of cells at the site of infection, as shown in \cite{Meszaros1999}; 
\item the coordination of macrophages and neutrophils to mount a more effective and intense response to LPS; 
\item the endothelium's dynamic permeability, which may depend on local concentrations of pro-inflammatory cytokines and protein granules; 
\item the important role of protein granules throughout the process of monocyte extravasation; 
\item the regulation of immune response by macrophages through the production of anti-inflammatory cytokines and the phagocytosis of 
apoptotic neutrophils; 
\item the crucial role of anti-inflammatory cytokines in the control of the inflammatory response, thus avoiding a state of persistent 
inflammation after the complete elimination of LPS. 
\end{itemize}

In future work, we plan to implement a more complete mathematical model that includes new cells (such as natural killer and dendritic 
cells), molecules and other processes involved in the immune response. The model could be extended by any of the following methods: 
a) including the interaction between endothelial cells, LPS and some cytokines such as IL-1$\beta$ and TNF-$\alpha$ \cite{Nooteboom2005}; 
b) incorporating the fact that high amounts of LPS also induce an increase in endothelium permeability \cite{Nooteboom2005}; 
c) considering the process of macrophage desensitization, in which high levels of LPS inhibit the production of TNF-$\alpha$ by 
macrophages \cite{Berg2001}; d) taking into account that the TNF-$\alpha$ produced by macrophages induces the production of even 
more TNF-$\alpha$ \cite{Sompayrac2008}; and e) considering that the TNF-$\alpha$ has proapoptotic and antiapoptotic effects on 
macrophages and neutrophils. In low concentrations, TNF-$\alpha$ delays the apoptosis of macrophages and neutrophils and induces 
the production of pro-inflammatory cytokines, whereas in high concentrations, it induces apoptosis \cite{Berg2001}.

An important final step is the validation of our proposed model using experimental data. 
Of particular interest is the spatio-temporal modeling of microabscess formation, a very important research topic. 
For instance, \cite{9284178, rigothier2002fat,21151499,10469395}  presents animal studies detailing the formation of liver abscess and 
microabscess by different types of infections. Epidermal microabscess formation by neutrophils was also evaluated 
in \cite{10771480, 16374458, 18006666} and \cite{Movat1986}. Infection of the heart by bacteria (bacterial myocarditis \cite{20937752}) 
or by viruses (viral myocarditis \cite{15668804}) is also correlated with microabscess formation by neutrophils. The interaction between 
tumor cells and inflammatory cells plays an important role in cancer initiation and progression and was investigated in \cite{21822201}
for the case of tumor-infiltrating neutrophils in pancreatic neoplasia, where the pattern of microabscess formation by neutrophils was 
reported once again. We acknowledge that this distinct pattern of formation can only be numerically reproduced and studied by models that 
capture the spatio-temporal dynamics of the HIS. Therefore, in the near future, we plan to extend our PDE model and adjust its parameters 
in the hopes of reproducing some of these experimental findings. 

%The authors declare that they have no competing interests

\bigskip

%%%%%%%%%%%%%%%%%%%%%%%%%%%%%%%%
\section*{Author's contributions}

RWS and ML have defined the methods and experiments. ABP has written the software code to implement the model 
and has performed all simulations. ABP and GCM have analyzed and interpreted the results. All authors have 
written the paper. They have read and approved the final manuscript. 


%%%%%%%%%%%%%%%%%%%%%%%%%%%%%%%%
\section*{Competing interests}    
The authors declare that they have no competing interests.

%%%%%%%%%%%%%%%%%%%%%%%%%%%

\section*{Acknowledgments}
\ifthenelse{\boolean{publ}}{\small}{}
The authors would like to thank FAPEMIG, CNPq, CAPES and UFJF for supporting this study in addition to the anonymous reviewers who 
have helped to improve the quality of this work. 



%%%%%%%%%%%%%%%%%%%%%%%%%%%%%%%%%%%%%%%%%%%%%%%%%%%%%%%%%%%%%
%%                  The Bibliography                       %%
%%                                                         %%              
%%  Bmc_article.bst  will be used to                       %%
%%  create a .BBL file for submission, which includes      %%
%%  XML structured for BMC.                                %%
%%  After submission of the .TEX file,                     %%
%%  you will be prompted to submit your .BBL file.         %%
%%                                                         %%
%%                                                         %%
%%  Note that the displayed Bibliography will not          %% 
%%  necessarily be rendered by Latex exactly as specified  %%
%%  in the online Instructions for Authors.                %% 
%%                                                         %%
%%%%%%%%%%%%%%%%%%%%%%%%%%%%%%%%%%%%%%%%%%%%%%%%%%%%%%%%%%%%%

\newpage
{\ifthenelse{\boolean{publ}}{\footnotesize}{\small}
 \bibliographystyle{bmc_article}  % Style BST file
  \bibliography{bibtex/relatorio} }     % Bibliography file (usually '*.bib' ) 

%%%%%%%%%%%

\ifthenelse{\boolean{publ}}{\end{multicols}}{}

%%%%%%%%%%%%%%%%%%%%%%%%%%%%%%%%%%%
%%                               %%
%% Figures                       %%
%%                               %%
%% NB: this is for captions and  %%
%% Titles. All graphics must be  %%
%% submitted separately and NOT  %%
%% included in the Tex document  %%
%%                               %%
%%%%%%%%%%%%%%%%%%%%%%%%%%%%%%%%%%%

%%
%% Do not use \listoffigures as most will included as separate files

\section*{Figures}

\subsection*{Figure 1 - Two-domain model}
Schematic representation of the two-domain model. The extravasation of neutrophils (or macrophages) from blood 
to tissue depends on local permeability, $P(x,t)$, and on the difference between local concentrations of neutrophils in the two domains, 
$N_{max}(x,t) - N(x,t)$. The permeability of the endothelium, which separates the two domains, varies with regard to time and space and depends 
on the local presence of pro-inflammatory cytokines and protein granules.

\subsection*{Figure 2 - Model's relations}
Relations among the components of the model.

\subsection*{Figure 3 - Diffusion process}
Schematic representation of the diffusion process.

\subsection*{Figure 4 - Permeability}
Representation of the differences between fixed and dynamic permeabilities.

\subsection*{Figure 5 - Chemotaxis process}
Schematic representation of chemotaxis.

\subsection*{Figure 6 - LPS concentration in space}
Temporal evolution of the spatial distribution of LPS.

\subsection*{Figure 7 - Neutrophil concentration in space}
Temporal evolution of the spatial distribution of neutrophil.

\subsection*{Figure 8 - Pro-inflammatory cytokine concentration in space}
Temporal evolution of the spatial distribution of pro-inflammatory cytokine.

\subsection*{Figure 9 - Active macrophage concentration in space}
Temporal evolution of the spatial distribution of active macrophage.

\subsection*{Figure 10 - Temporal evolution LPS}
Temporal evolution of the total quantity of LPS.

\subsection*{Figure 11 - Temporal evolution resting macrophage}
Temporal evolution of the resting macrophage population.

\subsection*{Figure 12 - Temporal evolution active macrophage}
Temporal evolution of the active macrophage population.

\subsection*{Figure 13 - Temporal evolution pro-inflammatory cytokine}
Temporal evolution of the pro-inflammatory cytokine population.

\subsection*{Figure 14 - Temporal evolution neutrophil}
Temporal evolution of the neutrophil population.

\subsection*{Figure 15 - Temporal evolution apoptotic neutrophil}
Temporal evolution of the apoptotic neutrophil population.

\subsection*{Figure 16 - Temporal evolution protein granule}
Temporal evolution of the protein granule population.

\subsection*{Figure 17 - Temporal evolution anti-inflammatory cytokine}
Temporal evolution of the anti-inflammatory cytokine population.

%%%%%%%%%%%%%%%%%%%%%%%%%%%%%%%%%%%
%%                               %%
%% Tables                        %%
%%                               %%
%%%%%%%%%%%%%%%%%%%%%%%%%%%%%%%%%%%

%% Use of \listoftables is discouraged.
%%
%\section*{Tables}
%  \subsection*{Table 1 - Sample table title}
%    Here is an example of a \emph{small} table in \LaTeX\ using  
%    \verb|\tabular{...}|. This is where the description of the table 
%    should go. \par \mbox{}
%    \par
%    \mbox{
%      \begin{tabular}{|c|c|c|}
%        \hline \multicolumn{3}{|c|}{My Table}\\ \hline
%        A1 & B2  & C3 \\ \hline
%        A2 & ... & .. \\ \hline
%        A3 & ..  & .  \\ \hline
%      \end{tabular}
%      }



%%%%%%%%%%%%%%%%%%%%%%%%%%%%%%%%%%%
%%                               %%
%% Additional Files              %%
%%                               %%
%%%%%%%%%%%%%%%%%%%%%%%%%%%%%%%%%%%

%\section*{Additional Files}
%  \subsection*{Additional file 1 --- Sample additional file title}
%    Additional file descriptions text (including details of how to
%    view the file, if it is in a non-standard format or the file extension).  This might
%    refer to a multi-page table or a figure.

%  \subsection*{Additional file 2 --- Sample additional file title}
%    Additional file descriptions text.


\end{bmcformat}
\end{document}







