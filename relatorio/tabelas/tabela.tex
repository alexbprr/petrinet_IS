\begingroup
	\renewcommand{\arraystretch}{1.5} % Default value: 1
	\begin{center}
		\def\arraystretch{2}
		\begin{longtable}{
				>{\centering\arraybackslash}m{2cm}|
				m{4.23cm}|
				>{\centering\arraybackslash}m{3cm}|
				m{3cm}|
				>{\centering\arraybackslash}m{2.5cm}
			}
			
			\caption{Transições}
			\label{tab:tabletransitions}\\
			
			\textbf{Nome} & \textbf{Descrição} & \textbf{Efeito} & \textbf{Places} & \textbf{Função}\\ \hline
			\endfirsthead
			
			\multicolumn{5}{c}{{\bfseries \tablename\ \thetable{} -- continuando da página anterior}} \\
			\textbf{Nome} & \textbf{Descrição} & \textbf{Efeito} & \textbf{Places} & \textbf{Função}\\
			\endhead
			
			\multicolumn{5}{r}{{Continua na próxima página}} \\
			\endfoot
			\hline \hline
			\endlastfoot
			
			Replication & 
			As bactérias se replicam com determinada probabilidade e a replicação possui uma saturação. & 
			Aumenta o número de tokens de B em 1. & 
			\parbox{3cm}{B [arc in = read, \\\ arc out = std]} & 
			$\frac{0.5 \times B}{B + 1}$ \\ \hline 
			
			Phag\_B\_N & 
			Modela a fagocitose das bactérias realizada pelos neutrófilos. & 
			Diminui o número de tokens de B em 1. & 
			\parbox{3cm}{B [arc in = std],\\\ N [arc in = read]\\\ D [arc in = mod]} & 			
			$\frac{0.1 \times B \times N}{0.5 \times B + 0.1 \times D + 1}$ \\ \hline
			
			Phag\_B\_M & 
			Modela a fagocitose das bactérias realizada pelos macrófagos.  & 
			Diminui o número de tokens de B em 1.  & 
			\parbox{3cm}{B [arc in = std]\\\ M [arc in = read]\\\ D [arc in = mod]} & 
			$\frac{0.05 \times B \times M}{0.5 \times B + 0.1 \times D + 1}$ \\ \hline
			
			
			N\_Death\_B & Modela a morte de neutrófilos causada pelas bactérias.  & 
			Diminui o número de tokens de N em 1.  & 
			\parbox{3cm}{B [arc in = read]\\\ N [arc in = std]\\\ ND [arc out = std]} & 
			$0.005 \times B \times N$ \\ \hline
			
			%M\_Death\_B & 
			%Modela a morte de macrófagos causada pelas bactérias.  & 
			%Diminui o número de tokens de M em 1.  & 
			%B [arc in = read] e M [arc in = std] & 
			%$0.0005 B M$ \\ \hline
			
			Inflam\_Mig & 
			Modela a migração de neutrófilos e macrófagos como consequência da inflamação causada pela morte de células do sistema imune e pelo dano tecidual.  & 
			Aumenta em 1 o número de tokens de N e M. & 
			\parbox{3cm}{D[arc in = read]\\\ AC[arc in = mod]\\\ N[arc out = std]\\\ M[arc out = std]} & 
			$0.005 \times D$ \\\hline
			
			N\_Migration1 & 
			Modela a migração dos neutrófilos atraídos pelas bactérias. 			
% 			devido a ação de citocinas pró-inflamatórias que não são modeladas explicitamente. 
% 			Essa transição simula a produção de citocinas pró-inflamatórias por outras células 
% 			(células de tecido e outras células do sistema imune não consideradas no modelo). 
			& 
			Aumenta o número de tokens de N em 1.  & 
			\parbox{3cm}{B[arc in = read]\\\ AC[arc in = mod] \\\ N [arc out = std] } & 
			$\frac{0.1 \times B}{0.1 \times B + 0.3 \times AC + 1}$ \\ \hline
						
% 			N\_Migration2 & 
% 			Modela a migração dos neutrófilos atraídos pelos macrófagos. & 
% 			Aumenta o número de tokens de N em 1.  & 
% 			\vspace{-0.41cm} \parbox{3cm}{B [arc in = read]\\\  N\_Mig2\_Aux \\\ [arc in = std]\\\ N [arc out = std]} \vspace{0cm} & 
% 			$\frac{0.05 N\_Mig2\_Aux}{(0.5 B + 1)}$ \\ \hline
			
			M\_Migration1 & 
			Modela a migração dos macrófagos atraídos pela bactéria. & 
			Aumenta em 1 o número de tokens de M. & 
			\parbox{3cm}{B[arc in = read]\\\ AC[arc in = mod]\\\ M[arc out = std]} & 
			$\frac{0.001 \times B + 0.000001}{0.1 \times AC + 1}$ \\\hline
			
			M\_Migration2 & 
			Modela a migração dos macrófagos atraídos pelos neutrófilos. & 
			Aumenta em 1 o número de tokens de M. & 
			\parbox{3cm}{N[arc in = read]\\\ AC[arc in = mod]\\\ M[arc out = std]} & 
			$\frac{0.001 \times N}{0.1 \times AC + 1}$ \\\hline
			
			N\_Apoptosis & 
			Modela a apoptose dos neutrófilos. & 
			Diminui em 1 o número de tokens de N. & 
			\parbox{3cm}{N [arc in = std]\\\ ND[arc out = std]} & 
			$0.02 \times N$ \\ \hline
			
			M\_Apoptosis & Modela a apoptose dos macrófagos. & 
			Diminui em 1 o número de tokens de M. & 
			\parbox{3cm}{M [arc in = std]\\\ MD [arc out = std]} & 
			$0.001 \times M$ \\	\hline
			
			B\_Damage & 
			Modela o dano tecidual causado pelas bactérias. & 
			Aumenta em 1 o número de tokens de D. & 
			\parbox{3cm}{B [arc in = read]\\\ D [arc out = std]} & 
			$0.005 \times B$ \\ \hline
			
			N\_Damage & 
			Modela o dano tecidual causado pelos neutrófilos necróticos. & 
			Aumenta em 1 o número de tokens de D. & 
			\parbox{3cm}{ND [arc in = read]\\\ D [arc out = std]} & 
			$0.01 \times ND$ \\ \hline
			
			Phag\_D\_M & 
			Modela a fagocitose de células de tecido mortas realizada pelos macrófagos. & 
			Diminui em 1 o número de tokens de D. & 
			\parbox{3cm}{D [arc in = std]\\\ M [arc in = read] \\\ AC\_Aux[arc out=std]} & 
			$0.005 \times M \times D$ \\ \hline
			
			Phag\_ND\_M & 
			Modela a fagocitose de neutrófilos apoptóticos realizada pelos macrófagos. & 
			Diminui em 1 o número de tokens de ND. & 
			\parbox{3cm}{ND [arc in = std]\\\ M [arc in = read] \\\ AC\_Aux[arc out=std]} & 
			0.05 $\times$ M $\times$ ND \\ \hline
			
			AC\_Prod & 
			Modela a produção de citocinas anti-inflamatórias. & 
			Aumenta em 1 o número de tokens de AC. & 
			\parbox{3cm}{AC\_Aux[arc in = std] \\\ AC[arc out=std]} & 
			$0.05 \times AC\_Aux$ \\\hline
			
			AC\_Decay & 
			Modela o decaimento de citocinas anti-inflamatórias. & 
			Diminui em 1 o número de tokens de AC. & 
			\parbox{3cm}{AC[arc in = std]} & 
			$0.05 \times AC$ \\
			
		\end{longtable}
	\end{center}
\endgroup


% \begin{sidewaystable}
%     \centering
% \caption{Wide table}
%     \label{tab:wide-item-tbl}
% \begin{tabularx}{\textwidth}{|*{4}{>{\RaggedRight\arraybackslash}X|}}
%     \hline
% \lipsum[1]  &   \lipsum[2]  &   \lipsum[3]  &   \lipsum[4]  \\
% \end{tabularx}
% \end{sidewaystable}

% \begin{longtable}{@{*}r||p{1in}@{*}}
% KILLED & LINE!!!! \kill
% \caption{Transições\label{long}}\\
% \hline\hline
% \multicolumn{2}{@{*}c@{*}}%
% {This part appears at the top of the table}\\
% \textsc{First}&\textsc{Second}\\
% \hline\hline
% \endfirsthead
% \caption[]{(continued)}\\
% \hline\hline
% \multicolumn{2}{@{*}c@{*}}%
% {This part appears at the top of every other page}\\
% \textbf{First}&\textbf{Second}\\
% \hline\hline
% \endhead
% \hline
% This goes at the&bottom.\\
% \hline
% \endfoot
% \hline
% These lines will&appear\\
% in place of the & usual foot\\
% at the end& of the table\\
% \hline
% \endlastfoot
% longtable  columns  are specified& in the \\
% same way as  in the tabular& environment.\\
% ...
% %\multicolumn{2}{||c||}{This is a ...}
% ...
% Some lines may take...&
% \raggedleft This last column is a ‘‘p’’ column...
% \tabularnewline
% ...
% Lots of lines& like this.\\
% ...
% \hline
% Lots\footnote{...} of lines& like this.\\
% Lots   of   lines& like this\footnote{...}\\
% \hline
% Lots of lines& like this.\\
% ...
% \end{longtable}

% \begin{tabulary}{\textwidth}{LCLLL}
%     \hline
%     \textbf{Name} & \textbf{Description} & \textbf{Effect} & \textbf{Places involved} & {Function}\\      
%     \hline    
%     Replication & As bactérias se replicam com determinada probabilidade e a replicação possui uma saturação. & Increase the number of tokens in B by 1.  & B [input arc = read, output arc = standard] & $\frac{0.1 B}{B + 1}$ \\
%     \hline
% \end{tabulary} 
